
\documentclass{beamer}
\usepackage{amsmath}
\usepackage{graphicx}
\graphicspath{ {./images/} }

\title{Least squares}
\subtitle{Sample Subtitle}
\author{Juan V. Vía}
\institute{}
\date{\today}

%\usetheme{lucid}
\begin{document}
	\frame {
		\titlepage
	}
	\frame {
		\frametitle{Example}
		\framesubtitle{Showing why least squares}
		We have a variable $y$. We know that it's dependent of another variable $x$ in some
		way. But we don't know how, exactly.

		So we go to the field and measure certain points. Those that we can reach. Six of them.
		$$(2,5),(5,5),(7,8),(11,7),(14,9),(18,7)$$

		That is: on $x=2$ we measure $y=5$, on $x=5$ we measure $y=5$ again, but
		on $x=7$ we got $y=8$, and so on.

	}

	\frame {
		\frametitle{Example}
		\framesubtitle{Showing why least squares}
		Next step, obviously, is to plot these points.


	}


	\frame{
		
		
		% \[\frac{-b \pm \sqrt{b^2 - c}}{2a}\]

		\[\begin{bmatrix}
			2 & 1\\
			5 & 1\\
			7 & 1\\
			11 & 1\\
			14 & 1\\
			18 & 1
		\end{bmatrix}\]
	}
	\frame{
		\frametitle{Sample Page 2}
		\framesubtitle{An Example of Lists}
		\begin{itemize}
			\item 1
			\item 2
			\item 3
		\end{itemize}
	}
	\frame{
	    \frametitle{Paragraph Content}
	    This is a paragraph.
	}
\end{document}